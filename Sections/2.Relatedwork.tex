\chapter{Related Work}
\label{ch:relatedwork}
The PID issue has been widely studied in the field of computer vision and IoT by using various devices. In the field of computer vision, camera is the most popular device. Face recognition technologies are surveyed in \cite{zhao2003face}. Reference \cite{parkhi2015deep} focuses on how to collect a very large training dataset and build a very deep CNN model for face recognition, but training process is extremely computationally expensive. A hybrid RFID and computer vision system for localization and tracking of RFID tags is proposed in \cite{goller2014fusing}. Reference \cite{isasi2010location} presents a solution which combines RFID with object tracking through cameras. Reference \cite{germa2010vision} presents a fusion system consisting of an RFID reader and a camera crew on a mobile robot platform to track people. These works \cite{goller2014fusing},\cite{isasi2010location},\cite{germa2010vision} fuse data from camera and RFID, but their accuracy highly depends on the density of RFID antennas. Thus, they are not suitable for longer range PID. Reference \cite{munaro2014fast} proposes a fast multi-people tracking algorithm for service robots through RGB-D camera. In \cite{spinello2011people}, people detection is realized by dense depth data, called Histogram of Oriented Depths (HOD). 